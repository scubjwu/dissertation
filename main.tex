%
% File:     example.tex
%
% Document based on the template for thesis and dissertations
% developed by Steven White and Malcolm Hutson.
%
% Extensively modified by Adam Lewis (awlewis@cacs.louisiana.edu) to
% meet the UL Graduate School requirements as of Spring 2011. 
%
% Unless otherwise expressly stated, this work is licensed under the
% Creative Commons Attribution-Noncommercial 3.0 United States License. To
% view a copy of this license, visit
% http://creativecommons.org/licenses/by-nc/3.0/us/ or send a letter to
% Creative Commons, 171 Second Street, Suite 300, San Francisco,
% California, 94105, USA.
%
% THE SOFTWARE IS PROVIDED "AS IS", WITHOUT WARRANTY OF ANY KIND, EXPRESS
% OR IMPLIED, INCLUDING BUT NOT LIMITED TO THE WARRANTIES OF
% MERCHANTABILITY, FITNESS FOR A PARTICULAR PURPOSE AND NONINFRINGEMENT.
% IN NO EVENT SHALL THE AUTHORS OR COPYRIGHT HOLDERS BE LIABLE FOR ANY
% CLAIM, DAMAGES OR OTHER LIABILITY, WHETHER IN AN ACTION OF CONTRACT,
% TORT OR OTHERWISE, ARISING FROM, OUT OF OR IN CONNECTION WITH THE
% SOFTWARE OR THE USE OR OTHER DEALINGS IN THE SOFTWARE.

\documentclass[12pt]{report}	% The documentclass must be ``report''.

% Dissertation package style file.
\usepackage{template/uldiss}  		

% 
% Here is a collection of optional packages that can make your life
% far more pleasant while writing your thesis, prospectus, or
% dissertation.   You should tailor these to match the specific needs
% for your document.
%
\usepackage{amsmath,amsthm,amsfonts,amscd,amssymb} % Some packages to write mathematics.
\usepackage{eucal} % Euler fonts
\usepackage{verbatim} % Allows quoting source with commands.
% The listings package supports the pretty-printing of source code.
% The listings packages supports the most commonly used programming
% languages.
\usepackage{listings}
\lstloadlanguages{Matlab,C++,C,Pascal}
\lstset{
         basicstyle=\footnotesize\ttfamily, 
         %numbers=left,              
         numberstyle=\tiny,          
         %stepnumber=2,              
         numbersep=5pt,              
         tabsize=2,                  
         extendedchars=true,         
         breaklines=true,            
         keywordstyle=textbf,    
         stringstyle=\ttfamily, 
         showspaces=false,       
         showtabs=false,         
         xleftmargin=17pt,
         framexleftmargin=17pt,
         framexrightmargin=5pt,
         framexbottommargin=4pt,
         %backgroundcolor=\color{lightgray},
         showstringspaces=false  
 }
% The caption package is used for fancy formatting of figure, table, and
% other captions.  It is very useful when combined with the listings package.
\usepackage{caption}
\DeclareCaptionFont{white}{\color{white}}
\DeclareCaptionFormat{listing}{\colorbox[cmyk]{0.43, 0.35, 0.35,0.01}{\parbox{\textwidth}{\hspace{15pt}#1#2#3}}}
\captionsetup[lstlisting]{format=listing,labelfont=white,textfont=white, singlelinecheck=false, margin=0pt, font={bf,footnotesize}}
\usepackage{pdfpages}
%
% The graphicx package is the standard package for importing of graphics
% into LaTeX documents.   Note that we configure the package to, by
% default, look for PNG, JPEG, and PDF files in a sub-directory of the
% current directory.
\usepackage{graphicx}
\DeclareGraphicsExtensions{.png,.jpg,.pdf}
\graphicspath{{graphics/}}
%
% Use the subfig package for dealing with multiple part figures.
\usepackage[caption=false,labelfont=sf,textfont=sf,captionskip=5pt]{subfig}
% 
% The comment package is useful when you use EMACS for editing LaTeX
% documents.  The table editor in EMACS orgtbl-mode interfaces with this
% package for easy editing of tables using the org-mode table editing
% functions. 
\usepackage{comment}

%% imported packages
%%
\usepackage{xargs}

\usepackage{mathpartir}
\usepackage{mathtools}
\usepackage{lambda,cc}
\usepackage[noshare]{vlc}
\usepackage{hyperref}
\usepackage{xcolor}
\usepackage{calc}
%\usepackage[labelformat=simple]{subcaption}
%\renewcommand\thesubfigure{(\alph{subfigure})}

\usepackage{stmaryrd}
\setcounter{tocdepth}{3}
\usepackage{makecell}
%\usepackage{setspace}

\usepackage[inline]{enumitem}
\usepackage{epstopdf}
\usepackage{booktabs}

\newcommand{\pg}{\ensuremath{p}}
\newcommand{\pgi}{\ensuremath{\pg_i}}
\newcommand{\pgj}{\ensuremath{\pg_j}}
\newcommand{\pgk}{\ensuremath{\pg_k}}

\newcommand{\seqijk}{\ensuremath{\dots,\pgi,\dots,\pgj,\dots,\pgk,\dots}}
\newcommand{\seqij}{\ensuremath{\dots,\pgi,\dots,\pgj,\dots}}


\newcommand{\sseqijk}{\ensuremath{\pg_1,\dots,\pgi,\dots,\pgj,\dots,\pgk,\dots}}

\newcommand{\pe}{\ensuremath{E}}

\newcommand{\pga}{\ensuremath{p_1}}
\newcommand{\pgb}{\ensuremath{p_3}}
\newcommand{\pgc}{\ensuremath{p_6}}

\newcommand{\nlevs}{\ensuremath{n}}
\newcommand{\tc}[1]{\ensuremath{O(#1)}}

\newcommand{\ongoing}{{}}
\newcommand{\temp}[1]{\textcolor{green}{\textbf{#1}}}

\newcommand{\toolS}{SHErrLoc}
\newcommand{\toolM}{\textsc{Mycroft}}
\newcommand{\toolD}{DOMSTED}
\newcommand{\toolMin}{MinErrLoc}
\newcommand{\toolH}{Helium}
\newcommand{\toolSk}{Skalpel}
\newcommand{\toolCh}{Chameleon}
\newcommand{\toolCf}{CFT}

\newcommand{\cs}{\ensuremath{\mathcal S}}
\newcommand{\con}{\ensuremath{C}}

\newcommand{\tv}{\ensuremath{\alpha}}
\newcommand{\tvf}{\ensuremath{\alpha_1}}
\newcommand{\tvs}{\ensuremath{\alpha_2}}
\newcommand{\tvt}{\ensuremath{\alpha_3}}
\newcommand{\tvv}{\ensuremath{\beta}}
\newcommand{\tvvf}{\ensuremath{\beta_1}}

\newcommand{\cname}{category}
\newcommand{\cnames}{categories}
\newcommand{\Cname}{Category}


\newcommand{\typel}{\prog{location}}
\newcommand{\typet}{\prog{type}}
\newcommand{\typer}{\prog{reason}}
\newcommand{\typee}{\prog{expression}}
\newcommand{\typeT}{\prog{Type}}
\newcommand{\typeR}{\prog{Reason}}
\newcommand{\typeE}{\prog{Expression}}
\newcommand{\typeL}{\prog{Location}}

\newcommand{\std}{\textit{std}}

\newcommand{\progsq}[1]{\prog{\textquotesingle#1\textquotesingle}}
\newcommand{\progdq}[1]{\prog{"#1"}}
\newcommand{\subele}[2]{$#1_{#2}$}
\newcommand{\arr}{$\to$}
%\newcommmand{\arrtp}[2]{$#1 \to #2$}
\newcommand{\newCompiler}{\textsc{Learnskell}}
\newcommand{\lines}[2]{Lines (#1-#2) omitted for brevity}
\newcommand{\TypeDiff}{\textit{TypeDiff}}
\newcommand{\TopDiff}{\textit{TopDiff}}
\newcommand{\BDiff}{\textit{BDiff}}
\newcommand{\TopLevelDiff}{\textit{TopFuncDiff}}
\newcommand{\TopBracketDiff}{\textit{TopBracDiff}}
\newcommand{\FuncDiff}{\textit{FuncDiff}}
\newcommand{\BracketDiff}{\textit{BracDiff}}
\newcommand{\TypeDiffs}{\textit{TypeDiffs}}
\newcommand{\TopDiffs}{\textit{TopDiffs}}
\newcommand{\BDiffs}{\textit{BDiffs}}
\newcommand{\TopLevelDiffs}{\textit{\TopLevelDiff{s}}}
\newcommand{\FuncDiffs}{\textit{FuncDiffs}}
\newcommand{\BracketDiffs}{\textit{\BracketDiff{s}}}
\newcommand{\Unify}{\textit{Unify}}
\newcommand{\where}{\textit{where}}
\newcommand{\otherwise}{\textit{otherwise}}
\newcommand{\Similarity}{\textit{Similarity}}
\newcommand{\smallwedge}{\mathrel{\text{\raisebox{0.25ex}{\scalebox{0.8}{$\wedge$}}}}}
\newcommand{\smallvee}{\mathrel{\text{\raisebox{0.25ex}{\scalebox{0.8}{$\vee$}}}}}
\newcommand{\tempPercent}[1]{\textbf{\textcolor{red}{#1\%}}}
\newcommand{\Ifit}{\textit{If}}
\newcommand{\ifit}{\textit{if}}
\newcommand{\thenit}{\textit{then}}
\newcommand{\andit}{\textit{and}}
\newcommand{\is}{\textit{is}}
\newcommand{\form}{\textit{form}}
\newcommand{\FE}{\textit{FE}}
\newcommand{\data}{\mathcal{D}}
\newcommand{\Var}{\textit{Var}}
\newcommand{\M}{\mathcal{M}}
\newcommand{\E}{\mathcal{E}}
\newcommand{\X}{\mathcal{X}}
\newcommand{\N}{\mathcal{N}}
\newcommand{\tp}{t_p}
\newcommand{\tn}{t_n}
\newcommand{\fp}{f_p}
\newcommand{\fn}{f_n}
\newcommand{\pr}{\textit{pr}}
\newcommand{\re}{\textit{re}}

\renewcommand{\progind}{0pt}
\newcommand{\cL}{{\cal L}}
\newcommand{\parag}[1]{\medskip\noindent\textbf{#1}\ \ }
\newcommand{\smote}{SMOTE}
\newcommand{\studySubmission}{226}
\newcommand{\randomForest}{Random-Forest}
\newcommand{\benchf}{Haskell-1}
\newcommand{\benchs}{Haskell-2}
\newcommand{\bencht}{Haskell-3}
\newcommand{\benchl}{Haskell-3}
\newcommand{\years}{\benchf}
\newcommand{\yeart}{\benchs}
%
\newcommand{\pui}[2]{\ensuremath{\prog{#1} =^? \prog{#2}}}
%
\newcommand{\mui}[2]{\ensuremath{#1 =^? #2}}

\newcommand{\mt}{\ensuremath{\tau}}

\newcommand{\mtexp}{\ensuremath{\mt_{\textit{exp}}}}
\newcommand{\mtinf}{\ensuremath{\mt_{\textit{inf}}}}


\newcommand{\ualg}{\ensuremath{\mathcal U}}
\newcommand{\dalg}{\ensuremath{\mathcal D}}
\newcommand{\td}{\textit{TD}}
\newcommand{\subtt}{\ensuremath{\theta}}
\newcommand{\sieves}{\textit{TD}}

\newcommand{\nameva}{(a)}
\newcommand{\namevb}{(b)}
\newcommand{\namevc}{(c)}
\newcommand{\namevd}{(d)}
\newcommand{\nameve}{(e)}
\newcommand{\namevf}{(f)}
\newcommand{\namevg}{(g)}
\newcommand{\namevh}{(h)}
\newcommand{\namevi}{(i)}
\newcommand{\namevj}{(j)}
\newcommand{\namevk}{(k)}
\newcommand{\namevl}{(l)}

\newcommand{\level}{\textit{lev}}
\newcommand{\union}[2]{\ensuremath{#1 \cup #2}}
\newcommand{\idx}{\textit{idx}}
\newcommand{\depth}{\textit{depth}}
\newcommand{\length}{\textit{numArity}}

\newcommand{\mtf}{\ensuremath{\mt_1}}
\newcommand{\mts}{\ensuremath{\mt_2}}

\newcommand{\restrict}[2]{\ensuremath{#1|_{#2}}}

\newcommand{\nextIdx}[2]{\ensuremath{\textit{next}(#1,#2)}}

\newcolumntype{Y}{>{\centering\arraybackslash}X}

\newcommand{\intVal}{integral}
\newcommand{\IntVal}{Integral}

\newtheorem{theorem}{Theorem}
\newtheorem{definition}{Definition}

\newcommand{\trainset}{Year-03}
\newcommand{\evalset}{Year-02}

\newcommand{\location}{Location}
\newcommand{\reason}{Reason}
\newcommand{\correct}{Specific}
\newcommand{\concrete}{Concrete}
\newcommand{\Location}{Location}
\newcommand{\Reason}{Reason}
\newcommand{\Correct}{Specific}
\newcommand{\Concrete}{Concrete}
\newcommand{\newTool}{\textsc{MLPEx}}
















\sloppy
\sloppypar

\renewcommand{\progind}{0pt}
\newcommand{\cL}{{\cal L}}
\newcommand{\parag}[1]{\medskip\noindent\textbf{#1}\ \ }
%% end importing
%%

%\usepackage{draftwatermark}	% Uncomment this line to have the
				% word, "DRAFT," as a background
				% "watermark" on all of the pages of
				% of your draft versions. When ready
				% to generate your final copy, re-comment
				% it out with a percent sign to remove
				% the word draft before you run
				% latex for the last time.
%
% Document Type
%
% Choose a document type by commenting out every other type.
%\prospectus
\dissertation
%\masterthesis
%\masterreport


%
% School Customizations 
%
% Select your school
% If your school is not listed, this template has not been specifically  
%   customized for you yet, but the Grad School requirements will be met.
%   Try one that might be similar.

\cacscmps
%   ACM Transactions bibliography

%\cacseecs
%   ACM Transactions bilbiography

%\schoolofmusic
%   Chicago style bibliography with footnotes?

%
% Basic Information
%

\author{Baijun Wu}
% Your name how it should normally appear across the document.
% The graduate school requires that your name always appear identically
%   every time that it is used. To help, we recommend use \theauthor
%   wherever your name should be printed for consistency.

\properauthor{Baijun Wu}
% Your proper name for alphabetizing. (Used in the abstract.)
% Last, First Middle Suffix

\title{Towards To User Friendly Error Debugging}
% The title of your thesis/dissertation. Use a tilde (~) for any
%   spaces that should not be broken at line breaks.

\dean{C. E. Palmer}
 % The Dean of the Graduate School

%\degree{Master of Science}
% The full title of your degree. 
% The default value is guessed by the document type and school.
  
%\major{Computer Science}
  % Your major.
  % The default value is guessed by your school.

%\graduationmonth{Spring}      
% Graduation semester, either Spring, Summer, or Fall, in the form
% as `\graduationmonth{Fall}'. Do not abbreviate.
% The default value (either Spring, Summer, or Fall) is guessed
% according to the time of running LaTeX.

% \graduationyear{2010} Graduation year, in the form as
% `\graduationyear{2001}'.  Use a 4 digit (not a 2 digit)
% number.  The default value is guessed according
%to the time of running LaTeX.
  
\previousdegrees{
Bachelor of Science, Sichuan University, 2008; 
Master of Science, Sichuan University, 2011; 
Doctor of Philosophy, University of Louisiana at Lafayette, \thegraduationmonth \ \thegraduationyear
}
% List all of your degrees, including the degree you are seeking with
% this document!

\abstractwordcount{120}
% The number of words in your abstract.
 % Unfortunately there is no clean way to count words in a section in Latex.
  
%
%
% Enter names of the member(s) of your committee. 
% Put one name per line with the name in square brackets. 
% The name on the last line, however, must be in curly braces.
%
% NOTE: The first member should be your supervisor.
%
% NOTE: Maximum six members. Minimum one member (supervisor).
%
\committeemembers
	[Erwin Schr\"odinger]
	{Albert Einstein}
	
\committeememberstitle
	[Professor of Physics]
	{Adjunct Professor of Math}

%\supervisortitle{Dr.}   
  % Your supervisor's title (Dr., Mrs., Mr., Sir, etc)
  %
  % The default value is "Dr."

%
% Change the	Hyphenation behavior.
%
%
\hyphenation{FORTRAN Hy-phen-a-tion}
% Manually specify how certain words should be hyphenated, if needed.
% You may add words without hyphens to request that they not be hyphenated.

%\hyphenpenalty=100000
% If you want no hyphenation in your document at all, uncomment
%   this line to set the hyphen penalty to an unreasonably
%   high value. 

%
% Some optional commands to change the document's defaults.
%
%
%\singlespacing
%\oneandonehalfspacing

%\singlespacequote
%\oneandonehalfspacequote

%\topmargin 0.125in	% Adjust this value if the PostScript file output
			% of your dissertation has incorrect top and 
			% bottom margins. Print a copy of at least one
			% full page of your dissertation (not the first
			% page of a chapter) and measure the top and
			% bottom margins with a ruler. You must have
			% a top margin of 1.5" and a bottom margin of
			% at least 1.25". The page numbers must be at
			% least 1.00" from the bottom of the page.
			% If the margins are not correct, adjust this
			% value accordingly and re-compile and print again.
			%
			% The default value is 0.125" 

	
%
%The document starts here.
%

\makeindex              % Make the index

\begin{document}

\titlepage              % Produces the title page.

\copyrightpage          % Produces the copyright page.

\approvalpage           % Produces the approval page

%
% Dedication, epigraph, and/or acknowledgments are optional, but must
% occur here.
%
%
\begin{dedication}
Dedicated to the people who I really care about.
\end{dedication}

\iffalse
\begin{epigraph}
If A is success in life,\\
then A equals x plus y plus z.\\
Work is x; y is play;\\
and z is keeping your mouth shut.\\
--Albert Einstein, \emph{Observer}, Jan. 15, 1950
\end{epigraph}
\fi

\begin{acknowledgments}		% Optional
Thank everyone....
\end{acknowledgments}

% Table of Contents will be automatically generated and placed here.
\tableofcontents   
% List of Tables will be placed here, if applicabl.e
\listoftables      
% List of Figures will be placed here, if applicabl.e
\listoffigures     

%
% Actual text starts here.%
%
% Including external files for each chapter makes this document simpler,
% makes each chapter simpler, and allows for generating test documents
% with as few as zero chapters (by commenting out the include statements).
% You can even change the chapter order by merely interchanging the order
% of the include statements.
%
%\include{chapter-introduction}

\chapter{Introduction}

In this dissertation, I present my research on type error debugging.
Understanding type error messages and fixing type errors is challenging for both novice and professional programmers.
Type errors can be caused for various reasons, for example, 
using wrong library functions,
using constants as functions, 
applying functions to arguments of wrong types, 
missing and having extra pairs of parentheses, and so on. 
In the first part of this dissertation, I present the insights about the type error debugging behaviors in practice, 
which are exploited to develop an effective error debugger in the later part of this dissertation.

This chapter motivates the needs of user friendly error messages
by investigating the challenges of debugging type errors.
It also outlines the structure of this dissertation and presents the contributions of this work.


\section{Motivation}
Type inference allows programs to be statically typed without
the presence of full type annotations. Most functional languages,
such as Haskell, ML, and OCaml support type inference, and many
imperative languages, such as C++, C\#, and Java, have started to
incorporate a limited form of type inference.
While type inference helps to save type annotations, learning languages
using type inference is quite challenging,
in particular for those who have background in imperative languages~\cite{clack1995dys,joosten1993teaching}.
Studies show that novice programmers tend to make type errors more often~\cite{chambers2012function,Heeren05:TQT,hage2006mining,tirronen2015understanding}, and
one reason may be that they have difficulties in learning modern type systems~\cite{clack1995dys,chakravarty2004risks}.


Understanding and fixing type errors is even harder~\cite{marceau2011measuring,marceau2011mind,tirronen2015understanding},
since type error messages generated by existing type checkers are usually ineffective~\cite{marceau2011mind}.
In particular, they may point to locations that are distant to real error causes,
expose errors in internal jargon, or provide misleading fixing suggestions.
Therefore, providing good quality type error messages
is important for beginners to study functional programming.

The problem of improving the quality of error messages has received
extensive attention. Many different approaches have been developed,
including error locating~\cite{Mcadam98:UST,Eo04:PSH,Zhang15:DTE,Pavlinovic14:FMT},
type error slicing for locating all the possible locations that
contribute to type errors~\cite{Schilling12:CFT,Haack03:TES},
inconsistency identification for finding program locations leading to
type conflicts~\cite{Yang00:ETE,Wazny06:TIT},
error explanation explaining why type errors occur and why certain
types are inferred~\cite{Chitil01:CET,jun2002explaining,Loncaric16:PFT},
error reparation that generates informative messages to fix type errors,
and interactive error debugging that allows users to move around
program ASTs and inspect the type of each node~\cite{Brassel04:TH,Chitil01:CET}.


While various methods have been proposed to locate
error causes more accurately and 
generate more informative change suggestions,
%improve the quality of error messages,
most of them work well under certain conditions.
Some methods~\cite{CE14popl,CE14flops,Zhang14:tgd}
work well when real error causes are
at leaves of ASTs. Others~\cite{Lerner06:SSM,Pavlinovic15:PST}
work well when there is only one type error but
not so well when there are multiple type errors.

%Therefore, we manually compute these statistical results and analyze them in our study.


\section{Research Goals}

So far, a good understanding about how type error debugging looks like
in practice is missing. As a result, it's unclear whether the conditions
for error debuggers to work well hold in practice or not.
%
There have been some efforts to collect relevant
information about errors made by students
~\cite{Hage09:Neon,tirronen2015understanding,chambers2012function,fenwick2009another,denny2012all}.
Neon~\cite{Hage09:Neon} is a domain specific language designed to query
program databases, which collected programs written by students learning Haskell.
It can extract various characteristics of the student programs,
for example, how the lengths of compiled modules evolve,
and how the average and median values of compilation intervals
change over a certain time period.
However, some information, like how far away is 
the error location
given by the type debugger from the real error cause,
can not be automatically generated by Neon.

A comprehensive overview of the mistakes made by beginners in Haskell was presented by \cite{tirronen2015understanding}.
The authors classified the errors made by students into three
categories: syntax errors, type errors, and run-time errors,
and then performed a fine-grained analysis for each category.
They showed some difficulties, like misuses of pattern matching,
in learning functional programming.
They also suggested that a more effective strategy of teaching type systems
is desirable.


While Neon and the overview study give some insights about kinds of errors the beginners of
functional programming made, they do not tell how errors were fixed and what students did.
%
This dissertation aims to address this problem by inspecting more than 2,700
ill-typed programs from 3 data sets,
recording various kinds of information about each type error, 
and analyzing the statistical results to extract high-level insights.
%
The empirical study results show that the type errors, arising from errors in grouping constructs like parentheses and brackets,
usually take more than 10 steps to fix and occur quite frequently in practice.
This class of errors is called as nontructural errors,
and existing error debuggers fail to generate precise and informative error messages for such errors.
In this dissertation, I will present a solution that delivers high quality error messages to fix nonstructural errors.


\section{Contributions and Outline of This Dissertation}

In this section, I present the structure of the remainder of this dissertation,
and along the way the contributions of this work are given.

Chapter~\ref{sec:review} (\emph{Literature Review}) collects research related to error debugging behaviors,  
type error debuggers and machine learning on programming languages.
This chapter contains material from \cite{wu2017type} and \cite{wu2017learning}.

Chapter~\ref{sec:background} (\emph{Background}) systematically explains under what conditions existing error debuggers work well
based on general ideas underlying many debuggers.
The result in this chapter is applicable to future debuggers that share the similar underlying ideas.
This chapter contains material from \cite{wu2017type}.

Chapter~\ref{sec:subjects} (\emph{Study Subjects}) presents the study subjects used in this work and
discuss the methodology used to study the process of fixing type errors based on the study subjects.
Five meaningful metrics are proposed to represent the real error debugging.
This chapter contains material from \cite{wu2017type}.

Chapter~\ref{sec:analysis} (\emph{Debugging Behavior Analysis}) shows the statistical results by analyzing the study subjects.
Many interesting observations that inform future research directions are derived.
This chapter contains material from \cite{wu2017type} and makes the following contributions.

\begin{enumerate}
\item The results in Section~\ref{sec:causes} show that about 45\% to 60\% of type errors are fixed by changing the structure of program ASTs 
and only about 22\% to 37\% of them are fixed by changing single leaves.
This indicates that most error debuggers won't work well in practice.

\item The results in Section~\ref{sec:annotation} show that on average about 30\% of type errors are caused by wrong type annotations.
This indicates that type annotations are unreliable for debugging ill-typed programs

\item The results in Section~\ref{sec:effectiveness} show that more concrete and precise error messages tend to be more effective for users to debug type errors.

\item The language features that cause type errors to be difficult to debug are presented in Section~\ref{sec:difficulty}.
The results indicate that function composition related operations (\prog{(.)}, \prog{\$} and parentheses),
point-free style function definitions, wrong type annotations, and wrong pattern matching often lead type error debugging to be challenging.
\end{enumerate}

Chapter~\ref{sec:features} (\emph{Nonstructural Type Error Representation}) presents the information extracted from error messages and program ASTs 
to represent nonstructural errors which are common in practice as shown in Chapter~\ref{sec:analysis}, but yet are handled poorly by existing debuggers.
Three kinds of information are considered: (1) the type conflicts, (2) the program structure around the error location,
and (3) the error messages from the underlying debugger.
This chapter contains material from \cite{wu2017learning} and makes the following contributions.

\begin{enumerate}
\item An algorithm for computing the differences between two types when they fail to unify is developed.
As error debuggers always use, among others, the heuristic of type difference of conflicting types to rank multiple error locations~\cite{Chen14:CFT,Hage07:HTE},
the proposed algorithm could be employed to develop more powerful error ranking heuristics.

\item In total 14 features are extracted to effectively represent nonstructural errors.
They provide a principled way to correlate type errors with program structures by covering a wide array of useful error information.
\end{enumerate}

Chapter~\ref{sec:solution} (\emph{Learning Nonstructural Errors}) provides the solution to generating user friendly error messages for nonstructural errors.
This capter contains material from \cite{wu2017learning} and makes the following contributions.

\begin{enumerate}
\item The motivation for using machine learning in developing an effective debugger for nonstructural errors is justified.

\item An algorithm for imbalanced classification problem is proposed to help determine if a type error is structural or nonstructural.

\item The evaluation results of the proposed solution are given, demonstrating that the machine learning-based error debugger is effective and scalable.
\end{enumerate}

Chapter~\ref{sec:conclusion} (\emph{Conclusion}) closes this dissertation with a summary of other applications of using machine learning in programming area,
the most important contributions of this work, and directions for future research.

\chapter{Literature Review}
\label{sec:review}

\section{Error Debugging Behaviors}
\label{sec:review:behavior}

\section{Type Error Debuggers}
\label{sec:reivew:debugger}

\subsection{Methods Handling Nonstructural Errors}

\subsection{Methods That Don't Handle Nonstructural Errors}

\section{Machine Learning on Programming Languages}
\label{sec:review:ml}


\chapter{Background}
\label{sec:background}

\section{When Errors Are Due to Single Leaves}
\label{sec:background:leaves}

\section{When Type Annotations Are Correct}
\label{sec:background:annotations}

\chapter{Study Subjects}
\label{sec:subjects}

\section{Analyzed Datasets}
\label{sec:subjects:db}

\section{Finding the Reference Program}
\label{sec:subject:ref}

\section{Analysis Metrics}
\label{sec:subject:metric}


\chapter{Debugging Behavior Analysis}
\label{sec:analysis}

\section{Overview of Error Debugging in Practice}
\label{sec:overview}

\section{Where Were Error Causes?}
\label{sec:causes}

\section{Were Type Annotations Reliable?}
\label{sec:annotation}

\section{When Are Error Message Effective?}
\label{sec:effectiveness}

\section{What Language Features Are Difficult?}
\label{sec:difficulty}

\chapter{Nonstructural Type Error Representation}
\label{sec:features}

\section{Unifying Non-Unifiable Types}
\label{sec:features:unify}

\section{The Feature Vector}
\label{sec:features:feature}

\chapter{Learning Nonstructural Errors}
\label{sec:solution}

\section{Motivation for Using Machine Learning}
\label{sec:solution:motivation}

\section{The Proposed Approach}
\label{sec:solution:approach}

\subsection{Preliminaries of Machine Learning}

\subsection{Data Preprocessing}

\subsection{Imbalanced Classification}

\subsection{Implementation}

\section{Evaluation}
\label{sec:solution:eval}


\chapter{Conclusion and Future Work}
\label{sec:conclusion}

\section{Other Applications}
\label{sec:conclusion:other}

\section{Main Contributions and Future Directions}
\label{sec:conclusion:close}

%\include{chapter11}	

%\include{chapter12}	
%
% Appendix/Appendices
%
% If you have only one appendix, use the command \appendix instead
% of \appendices.
%
%\appendices

%\include{chapter-appendix1}
%\include{chapter-appendix2}
%\include{chapter-appendix3}

%
% Generate the bibliography.
%

%\nocite{*}      % This command causes all items in the               %
                % bibliographic database to be added to              %
                % the bibliography, even if they are not             %
                % explicitly cited in the text.                      %

% Here the bibliography is inserted.
% Replace "example" with the name of your ".bib" file
\bibliography{error-reporting,me,paper}                      
\index{Bibliography@\emph{Bibliography}}
%


%
% Generate the index.
% 
%
\printindex     % Include the index here. Comment out this line      
%               % with a percent sign if you do not want an index .  
%

\begin{abstract}
test
\end{abstract}

\iffalse
\begin{biography}
  Lorem ipsum dolor sit amet, consectetur adipiscing elit. Duis
  tristique nibh nec enim egestas lobortis. Cum sociis natoque penatibus
  et magnis dis parturient montes, nascetur ridiculus mus. Praesent
  vitae lorem a ante congue imperdiet. Aenean gravida lacus ac sem
  pharetra facilisis laoreet elit facilisis. Integer ullamcorper blandit
  lorem, at pharetra felis vestibulum a. Cras vitae elit non neque
  lacinia euismod eget ac orci. Ut nibh ligula, porttitor eget facilisis
  id, fringilla eu sapien. Vivamus in congue dui. Nunc lacus nunc,
  ornare eu vulputate quis, auctor id lacus. Ut luctus interdum
  ligula. Donec ac rhoncus urna. Ut suscipit nulla molestie libero
  pretium sed lacinia elit commodo. Nulla a neque eget odio consequat
  bibendum. Aliquam egestas sollicitudin eros at sollicitudin. Proin et
  ipsum at dolor molestie rutrum vitae et leo.
\end{biography}
\fi

\end{document}
% The following comment block is used by the different flavors of EMACS and
% the AUCTEX package to manage multiple documents.  In order for AUCTEX
% to understand you're working with multiple files, you should define
% the TeX-master variable as a file local variable that identifies your
% master document.
%
% Please do not remove.
%%% Local Variables: 
%%% mode: latex
%%% TeX-master: "example.tex"
%%% End: 
